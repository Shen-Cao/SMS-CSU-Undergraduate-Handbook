\documentclass{book}
% \documentclass[12pt]{article}
\usepackage{graphicx} % Required for inserting images
\usepackage[a4paper, left=1in, right=1in, top=1in, bottom=1in]{geometry}
\usepackage{fancyhdr}
% \usepackage{fontspec}
\usepackage{listings}
\usepackage{ragged2e} % 用这个宏包与\justifying命令将所有文字两端对齐!
\usepackage{amsmath, amsthm}
\usepackage{amsfonts,amssymb}
\usepackage{mathrsfs}


\usepackage{subfigure}
\usepackage{enumerate}

% PDF超链接生成
\usepackage{hyperref} 

% 中文字体编译支持
\usepackage{ctex}

\usepackage{makecell}
\newcommand{\tabincell}[2]{\begin{tabular}{@{}#1@{}}#2\end{tabular}}
% Drawing
\usepackage{tikz}
\usetikzlibrary{positioning} % for setup of relative position
\usepackage{xcolor} % for color options

% 设置参考文献格式
\bibliographystyle{plain}

\title{中南大学数学与统计学院本科生手册}
\author{中南大学“数院之光”全体成员}
\date{July 2024}

\begin{document}

\maketitle

\hspace{64pt}
\begin{center}
\textit{可以在这里加个名言警句用来装逼}
\end{center}
\begin{flushright}
\textit{ - 名言警句的作者}
\end{flushright}



\pagestyle{fancy}
\fancyhead{}
\fancyhead[LE]{\textsl{\rightmark}}
\fancyhead[RO]{\textsl{\leftmark}}
\renewcommand{\headrulewidth}{1pt} %分隔线宽度4磅
\renewcommand{\footrulewidth}{0pt}

\setlength{\parskip}{.5em}
% \setlength{\parindent}{-20pt}
\setlength{\lineskiplimit}{2em}
\setlength{\lineskip}{2pt}
% \setlength{\linespread}{2em} % 设置行距
% \vspace{1em}

% \newpage
\tableofcontents
% \pagestyle{fancy}
% \newpage


% % % % % % % % % % % % % % % % 
\chapter{简介}

此部分对数院之光以及本手册作简单介绍。


\chapter{专业:数学与应用数学}


\chapter{专业:统计学}


\chapter{专业:信息与计算科学}







\chapter{通用信息}



% % % % % % % % % % % % % % % % 
\chapter{附录}

\section{文件编篡方案}

这部分明确一些文件编篡的标准。

\subsection{封面与目录}

暂时保持默认设置。

\subsection{标题}

暂时保持默认设置。

\subsection{正文}

暂时保持默认设置。

\subsection{图片与表格}

图片使用figure环境,如范例\ref{fig:template-0}。
\begin{figure}[htbp]
    \centering
    \includegraphics[width=0.4\linewidth]{../Appendices/Figures/数院之光-透明.png}
    \caption{图片范例}
    \label{fig:template-0}
\end{figure}

表格使用table环境,统一为三线表,如范例\ref{tab:template-0}。

\begin{table}[htbp]
    \centering
    \caption{表格范例}
    \begin{tabular}{ p{2cm} | p{4.5cm} | p{4.5cm} }
        表头 & A & B \\ \hline
        1 & A1 & B1 \\ 
        2 & A2 & B2 \\
        3 & A3 & B3
    \end{tabular}
    \label{tab:template-0}
\end{table}


\subsection{GitHub Repository的维护}



\subsection{参考文献范例}
这是参考文献的范例\cite{Einstein_Podolsky_Rosen_1935}. 



\bibliography{./bibliography.bib}


\end{document}
